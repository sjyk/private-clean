\section{Query Processing}
In the following analysis, we describe how to estimate \sumfunc, \avgfunc, and \countfunc
queries on cleaned private relations. 
Recall, that we support queries of the following form:
\begin{alltt}
SELECT \textsf{f}(a) FROM R WHERE cond(d)
\end{alltt}
As in the previous section, let $d$ be a member of $g_i$ which has $N$ be the number of distinct values, and $l$ be the number of queried attributes.

\subsection{COUNT Estimation}
We first start with \countfunc queries.
Let $c_{private}$ be the count on the cleaned GRR view.
Using the quantities described in the previous section, we can derive that the expected value of this count:
\[
\mathbb{E}(c_{private})=c_{true}\tau_p+(S-c_{true})\tau_n
\]
Intuitively, this is the true count $c_{true}$ multiplied by the true positive rate, plus the complement multiplied by the false positive rate.
Solving for $c_{true}$, we find that:
\[ c_{true} = \frac{\mathbb{E}(c_{private})-S\tau_n}{(\tau_p-\tau_n)} \] 
Using the properties of expectation, namely conditioning and linearity, we can show that this defines an estimator for $c_{true}$: 
\[ \widehat{c_{true}} = \mathbb{E}(c_{true} \mid c_{private}) = \frac{c_{private}-S\tau_n}{(\tau_p-\tau_n)} \] 
Furthermore, this estimator is \emph{unbiased} over all realizations of $c_{private}$.

\vspace{0.5em}

\noindent\textbf{Estimating Count: \textmd{$\hat{c}=\frac{c_{private}-S\tau_n}{(\tau_p-\tau_n)}$}}


\subsubsection{Finite-Sample Error}
Next, we analyze the error in this estimate using non-asymptotic analysis.
We prove a bound that is valid for all $l$; that is queries of all selectivities.
We are interested in providing the user a guarantee on the error for any GRR view derived from a dataset of large enough size.
To bound this estimate in confidence intervals, we can apply Hoeffding's inequality:
\[
P(\mid c_{private}-\mathbb{E}(c_{private})\mid\ge t)\le2\exp(-\frac{2t^{2}}{S})
\]
With a little bit of algebraic manipulation, we arrive at confidence intervals with probability $1-\alpha$:
\[
c_{true}=\widehat{c_{true}}\pm\frac{1}{1-p}\sqrt{\frac{S}{2}\ln(\frac{2}{\alpha})}
\]

%\subsubsection{Asymptotic Error}


\iffalse
\begin{theorem}
Given a COUNT query with a predicate that spans $l$ entities out of $N$ possible entities, the result estimate and the confidence intervals are:
\noindent\textbf{Confidence Intervals with probability $1-\alpha$:}
\[c_{true}=\hat{c}\pm\frac{1}{1-p}\sqrt{\frac{S}{2}\ln(\frac{2}{\alpha})}\]
\end{theorem}
\begin{proof}
To prove this, first, we formulate an unbiased estimator:
\[
\mathbb{E}(c_{private})=c_{true}\tau_p+(S-c_{true})\tau_n
\]
\[
\mathbb{E}(c_{private})=c_{true}(\tau_p-\tau_n)+S\tau_n
\]
\[
\frac{\mathbb{E}(c_{private})-S\tau_n}{(\tau_p-\tau_n)}=c_{true}
\]
Then, applying Hoeffding's inequality:
\[
P(\mid c_{private}-\mathbb{E}(c_{private})\mid\ge t)\le2\exp(-\frac{2t^{2}}{S})
\]
\[
\sqrt{\frac{S}{2}\ln(\frac{2}{\alpha})}=t
\]
It follows that for this estimator:
\[
c_{true}=\hat{c}_{true}\pm\frac{1}{1-p}\sqrt{\frac{S}{2}\ln(\frac{2}{\alpha})}
\]
\[
c_{true}=\hat{c}_{true}\pm\frac{\gamma_{\alpha}}{1-p}\sqrt{\frac{S}{2}}
\]
\end{proof}
\fi

\subsection{SUM Estimation}
Next, we analyze \sumfunc queries.
It is important to note that unlike \countfunc, the \sumfunc queries also query the numerical attributes with the Laplace randomization mechanism.
Like before, we first formulate an expression for the true sum $h_{private}$:
\[
\mathbb{E}(h_{private})=c_{true}\mu_{true}\tau_p+(S-c_{true})\mu_{false}\tau_n
\]
where $\mu_{true}$ the true mean value of the numerical attributes that satisfy the predicate.
The challenge is that there are two unknown variables $c_{true}\mu_{true}$
and $\mu_{false}$.
Thus, the problem is that there is not enough information in this equation to solve for $c_{true}\mu_{true}$.
So we can apply a trick where we also calculate the sum over the complement as well:
\[
\mathbb{E}(h_{private}^{c})=\gamma_n c_{true}\mu_{true}+(S-c_{true})\gamma_p\mu_{false}
\]
This is defines a linear system of equations, and solving it arrives at:
\[
h_{true} =\frac{(N-lp)\mathbb{E}(h_{private})-(lp)\mathbb{E}(h_{private}^{c})}{(1-p)N}
\]
Applying the same expectation trick as before, we can arrive at the following unbiased estimator:

\vspace{0.5em}

\noindent\textbf{Estimating Sum: \textmd{$\frac{(N-lp)h_{private}-(lp)h_{private}^{c}}{(1-p)N}$}}

\subsubsection{Finite-Sample Error}
Laplace random variables are a class of random variables called sub-gamma random variables.
Sums of these random variables can be bounded using Bernstein's inequality, resulting in the following bound:
\[
h_{true}=\widehat{h_{true}}\pm\frac{1}{(1-p)N}\cdot b(\sqrt{2S\ln(\frac{1}{\alpha})}+\ln(\frac{1}{\alpha}))
\]
In this bound, there is an explicit dependence on b which is the numerical randomization.

\subsection{AVG Estimation}
It turns out that due to a ratio of random variables problem, the intuitive estimator for \avgfunc $\frac{\widehat{h_{true}}}{\widehat{c_{true}}}$ is biased.
However, empirically, this bias is small and in fact such estimates are sometimes called conditionally unbiased \cite{DBLP:conf/sigmod/HellersteinHW97}.

\vspace{0.5em}

\noindent\textbf{Estimating Average: \textmd{$\frac{\widehat{h_{true}}}{\widehat{c_{true}}}$}}

\vspace{0.5em}

\subsubsection{Finite-Sample Error}
Confidence intervals can be computed directly from the expressions above by taking the upper confidence interval of $\widehat{h_{true}}$ and dividing it by the lower confidence interval of $\widehat{c_{true}}$, and vice versa.
To get an approximate analytic form for this interval, we can apply standard error propagation techniques:
\[
\delta_{avg} \approx \frac{1}{c_{true}} \cdot  \frac{\delta_{sum}}{\delta_{count}}
\]
where $\delta$ denotes the width of the confidence interval.
This analytic form shows that there is a very strong dependence on the selectivity of the query, namely $\frac{1}{c_{true}}$.

\subsection{Summary}
We derived query result estimators for \sumfunc, \avgfunc, and \countfunc queries on cleaned GRR views.
Our analysis reveals a couple of interesting insights.
First, the error in the query result is linear (as opposed to quadratic or exponential) in the privacy parameters $b$ and $\frac{1}{1-p}$.
One of the general criticisms of differential privacy is the difficulty of choosing the parameter $\epsilon$.
Our analysis suggests that in the special case of GRR, the tradeoff between privacy and utility is relatively well behaved.
As privacy increases, the query result gradually becomes more erroneous.
This makes it easy to invert these functions to set a desired privacy level.

\begin{example}
In the running example dataset, let us suppose $p$ is 0.25, there are 25 distinct majors in the dirty relation, and 500 records. We are interested in counting the number of engineering majors (account for 10 out of the initial 25). 
Suppose, the private count was $300$.
Using the formula described above $est = \frac{300-500\times10\times.25\times \frac{1}{25}}{.75}= 333.3$.
The 95\% confidence intervals are $ \pm \frac{1}{.75} \times \sqrt{250} \times \sqrt{\ln \frac 40}= \pm 41$.
\end{example}

\iffalse
We can apply the following trick

\begin{theorem}
Given a SUM query with a predicate that spans $l$ entities out of $N$ possible entities, the result estimate and the confidence intervals are:

\noindent\textbf{Estimating SUM: \textmd{$\hat{m}=\frac{(N-lp)m_{private}-(lp)m_{private}^{c}}{(1-p)N}$}}

\noindent\textbf{Confidence Intervals with probability $1-\alpha$:} 
\[m_{true}=\hat{m}_{true}\pm\frac{1}{(1-p)N}\cdot b_{i}(\sqrt{2S\ln(\frac{1}{\alpha})}+\ln(\frac{1}{\alpha}))\]
\end{theorem}
\begin{proof}
To prove this, first, we formulate an unbiased estimator:
\[
\mathbb{E}(m_{private})=c_{true}[(1-p)+p\frac{l}{N}]\mu_{true}+(S-c_{true})(p\frac{l}{N})\mu_{false}
\]
We are interested in estimating the count which is $c_{true}\mu_{true}$:
\[
\mathbb{E}(m_{private})=[(1-p)+p\frac{l}{N}]c_{true}\mu_{true}+(S-c_{true})(p\frac{l}{N})\mu_{false}
\]
The challenge is that there are two unknown variables $c_{true}\mu_{true}$
and $\mu_{false}$, so we can apply a trick where we also calculate
the sum over the complement and solve:
\[
\mathbb{E}(m_{private}^{c})=[p\frac{N-l}{N}]c_{true}\mu_{true}+(S-c_{true})(1-p+p\frac{N-l}{N})\mu_{false}
\]
This is a linear system of equations, and solving it arrives at:
\[
\begin{bmatrix}m_{private}\\
m_{private}^{c}
\end{bmatrix}=\begin{bmatrix}(1-p)+p\frac{l}{N} & (S-c_{true})(p\frac{l}{N})\\
p\frac{N-l}{N} & (S-c_{true})(1-p+p\frac{N-l}{N})
\end{bmatrix}\begin{bmatrix}c_{true}\mu_{true}\\
\mu_{false}
\end{bmatrix}
\]
\[
\begin{bmatrix}(1-p)+p\frac{l}{N} & (S-c_{true})(p\frac{l}{N})\\
p\frac{N-l}{N} & (S-c_{true})(1-p+p\frac{N-l}{N})
\end{bmatrix}^{-1}\begin{bmatrix}m_{private}\\
m_{private}^{c}
\end{bmatrix}=\begin{bmatrix}c_{true}\mu_{true}\\
\mu_{false}
\end{bmatrix}
\]
\[
\frac{1}{(S-c_{true})(1-p+p\frac{N-l}{N})[(1-p)+p\frac{l}{N}]-(S-c_{true})(p\frac{l}{N})p\frac{N-l}{N}}\begin{bmatrix}(S-c_{true})(1-p+p\frac{N-l}{N}) & -(S-c_{true})(p\frac{l}{N})\\
-p\frac{N-l}{N} & (1-p)+p\frac{l}{N}
\end{bmatrix}
\]
\[
\frac{1}{S-c_{true}}\cdot\frac{1}{(1-p+p\frac{N-l}{N})[(1-p)+p\frac{l}{N}]-(p\frac{l}{N})p\frac{N-l}{N}}\begin{bmatrix}(S-c_{true})(1-p+p\frac{N-l}{N}) & -(S-c_{true})(p\frac{l}{N})\\
-p\frac{N-l}{N} & (1-p)+p\frac{l}{N}
\end{bmatrix}
\]
\[
\frac{1}{S-c_{true}}\cdot\frac{1}{1-p}\begin{bmatrix}(S-c_{true})(1-p+p\frac{N-l}{N}) & -(S-c_{true})(p\frac{l}{N})\\
-p\frac{N-l}{N} & (1-p)+p\frac{l}{N}
\end{bmatrix}\begin{bmatrix}m_{private}\\
m_{private}^{c}
\end{bmatrix}=\begin{bmatrix}c_{true}\mu_{true}\\
\mu_{false}
\end{bmatrix}
\]
\[
c_{true}\mu_{true}=\frac{(1-p+p\frac{N-l}{N})m_{private}-p\frac{l}{N}m_{private}}{1-p}
\]
With a little bit of algebra, we can arrive at:
\[
c_{true}\mu_{true}=\frac{(N-lp)m_{private}-(lp)m_{private}^{c}}{(1-p)N}
\]
Applying Hoeffding's inequality:
\[
m_{true}=\hat{m}_{true}\pm\frac{1}{(1-p)N}\cdot b_{i}(\sqrt{2S\ln(\frac{1}{\alpha})}+\ln(\frac{1}{\alpha}))
\]
\end{proof}
\fi