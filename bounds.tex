\section{$\alpha$ Error Detectability}
We first present analysis about the privacy mechanism itself, and how successfully an analyst can detect errors in a private relation.



To analyze this, we first consider a lemma about the domain of a single attribute group:
\begin{lemma}
For a dataset size $S>\frac{N}{1-p}\ln(\frac{1}{\alpha})$, with probability
$1-\alpha$ the domain is preserved. First, let us start with the
probability that one value $x$ is not preserved:
\[
P[x\text{ is masked}]=\alpha=[\frac{p(N-1)}{N}]^{S}
\]
\end{lemma}
\begin{proof}
If we solve for $S$, apply the inequalities that $\frac{N-1}{N}\le1$
and $ln(x)\le x-1$, we get
\[
\frac{\ln(\frac{1}{\alpha})}{1-p}\le S
\]
Then applying a union bound we get:
\[
\frac{N}{1-p}\ln(\frac{1}{\alpha})\le S
\]
\end{proof}

Applying a union bound over the lemma we get:
\begin{theorem}
A dataset size of $S$ greater than:
\[
S\ge(\sum_{i}\frac{N_{i}}{1-p_{i}})\ln(\frac{1}{\alpha})
\]
ensures $\alpha$-error detectability.
\end{theorem}
\begin{proof}
We apply a union bound to arrive at the formula:
\[
S\ge(\sum_{i}\frac{N_{i}}{1-p_{i}})\ln(\frac{1}{\alpha})
\]
\end{proof}

We can use this analysis to determine the maximum value of $p$ that allows for $\alpha$ error detectability:
\begin{corollary}
For a desired domain preservation probability $1-\alpha$, dataset size $S$,
the maximum randomized response privacy parameter is $p_{max}=1-\frac{(1-\alpha)\sum_{i}N_{i}}{S}$.
\end{corollary}
\begin{proof}
\[
\frac{\sum_{i}N}{1-p_{max}}\ln(\frac{1}{\alpha})\ge(\sum_{i}\frac{N_{i}}{1-p_{i}})\ln(\frac{1}{\alpha})
\]
\[
\frac{\sum_{i}N}{1-p_{max}}\ln(\frac{1}{\alpha})=S
\]
\[
\frac{1}{1-p_{max}}=\frac{S}{\ln(\frac{1}{\alpha})\sum_{i}N}
\]
\[
p_{max}\ge1-(1-\alpha)\frac{\sum_{i}N}{S}
\]
\end{proof}
